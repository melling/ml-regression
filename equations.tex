\documentclass[fleqn,9pt]{article}

%\documentclass[a4paper]{article}
%
% https://blmoistawinde.github.io/ml_equations_latex/

\usepackage{amsmath,amssymb}

%\usepackage{multicol}

\usepackage{titling}
\date{} % no date

% https://kb.mit.edu/confluence/pages/viewpage.action?pageId=3907057
\usepackage[margin=0.1in]{geometry}

%\geometry{
%    margin=0.55in,
%    top=0.35in,
%    bottom=0.35in,
%}
 
\usepackage{tikz}
\usetikzlibrary{
    arrows,
    arrows.meta,
    positioning,
}

\setlength{\parindent}{0pt}
\setlength{\parskip}{1\baselineskip}
\setlength{\mathindent}{0pt} % fleqn option needed

% https://tex.stackexchange.com/questions/24561/setting-the-column-gap-in-a-twocolumn-or-multicol-document
%\setlength{\columnsep}{3cm}

\setlength{\droptitle}{-5em}   % This is your set screw

\newcommand{\indep}{\perp \!\!\! \perp}

\title{Machine Learning Advanced Topics Group Formulas \vspace{-6em}}

%\pagestyle{empty}

\begin{document}
\maketitle


% Basic use of minipage environment

%\noindent
\begin{minipage}[t]{0.33\textwidth}

\textbf{Linear Algebra}

\begin{equation*}
\begin{split}
	A(B+C) = AB + AC\\
	A(BC) = (AB)C \\
	(AB)^T =B^TA^T\\
	A^TA = AA^T = I \implies A^T = A^{-1}\\
	x^Ty = y^Tx\\
		x^Tx - inner\\
	xx^T - outer\\
\end{split}	
\end{equation*}

\begin{multline*}
\Phi: V \to W \\
\Phi(x+y) = \Phi(x) + \Phi(y)\; (2.85) \\
	\Phi(\lambda x) =  \lambda \Phi(x)\\
\forall x,y\in V, \forall \lambda, \psi \in \mathbb{R} : \Phi(\lambda x + \psi y) 
\end{multline*}

\begin{equation*}
P_\pi=\frac{bb^T}{\|b\|^2} (3.46) \\
\end{equation*}

\begin{multline*}
ker(\Phi) = \Phi^{-1}(0_W) 
 = \{v\in V: \Phi(v) =0\} (2.122)\\
Im(\Phi) = \Phi(V) 
 =  \{w \in W | \exists v\in V: \Phi(v) = w \}\\
\end{multline*}

\begin{equation*}
\begin{split}
&\|x\|_1 := \sum_i^n |x_i| =\ell_1 \;(3.3)\\
&\|x\|_2 := \sqrt{\sum_i^n x_i^2} = \sqrt{\textbf{x}^T \textbf{x}} =\ell_2 \;(3.4)\\
&\forall x \in V\setminus\{0\}: x^TAx > 0\;(3.11)\\
&\|x\| := \sqrt{\langle x,x\rangle}\;(3.16)\\
&A = P D P^{-1}\;(4.55)\\
&A^k =(P D P^{-1})^k = P D^k P^{-1}\;(4.62)\\
&A = U\Sigma V^T\\
\end{split}	
\end{equation*}

\textbf{Probability}
%\begin{center}
\begin{equation*}
\begin{split}
	p(x,y) = p(x)p(y) -  x \indep y\\
	p(x|y) = p(x) -  x \indep y\\
	\mathbb{V}_{X,Y} =  \mathbb{V}_X[x] + \mathbb{V}_Y[y] - x \indep y\\
	Cov(x,y) = 0 - x \indep y\\
% Product Rule (6.22) p184
p(x,y) = p(y|x)p(x) - (6.22)\\
p(x|y) = \frac{p(y|x)p(x)}{p(y)} - (6.23)\\
Cov[x,y] = \mathbb{E}[xy]-\mathbb{E}[x]\mathbb{E}[y]\;(6.36)\\
corr[x,y] = \frac{Cov[x,y]}{\sqrt{\mathbb{V}[x] \mathbb{V}[y]}}\;(6.40)\\
X\sim \mathcal{N}(\mu,\sigma^2)\\
X\sim \mathcal{N}(\mu, \boldsymbol{\Sigma^2})\\
X \sim U(0, 1)\\
\end{split}		
\end{equation*}
%\end{center}

\end{minipage}% <---------------- Note the use of "%"
\begin{minipage}[t]{0.33\textwidth}



\textbf{Optimization}
% Jensen's Inequality (7.30) p236
\begin{equation*}
	f(\theta x + (1-\theta)y) = \theta f(x) + (1-\theta)f(y)
\end{equation*}

\textbf{Linear Regression}

\begin{equation*}
\begin{split}
	\hat{y} = \beta_0 + \beta_1 x_1 \cdots + \beta_n x_n \\
	\boldsymbol{Y} = \boldsymbol{x}^T \boldsymbol{\beta} \\
   \ell_{L1} = Error(Y - \widehat{Y}) + \lambda \sum_1^n |w_i|\\
   \ell_{L2} = Error(Y - \widehat{Y}) +  \lambda \sum_1^n w_i^{2}\\
\end{split}		
\end{equation*}

\textbf{Logistic Regression}

\begin{equation*}
odds = \frac{p(X)}{1-p(X)}
\end{equation*}

% ISLR (4.2))

\begin{equation*}
	\phi(x) = \frac{1}{1+e^{-x}}
\end{equation*}


\begin{equation*}
	p(x) = \frac{e^{\beta_0 + \beta_1 x_1 \cdots + \beta_m x_p}}{1+e^{\beta_0 + \beta_1 x_1 \cdots + \beta_m x_p}}
\end{equation*}

\begin{equation*}
\begin{split}
\text{BCE} =& -(y\,log(p)+(1-y)log(1-p))\\
\text{MLE} =& \max_{\theta} \prod_y p(y;\theta)
\end{split}
\end{equation*}


\begin{equation*}
\begin{split}
&\text{Accuracy} = \frac{TP+TN}{TP+TN+FP+FN}\\
&\text{Precision} = \frac{TP}{TP+FP}\\
&\text{Recall} = \frac{TP}{TP+FN}\\
&\text{Sensitivity} = Recall = \frac{TP}{TP+FN}\\
&\text{Specificity} = \frac{TN}{FP+TN}\\
&\text{F1} = \frac{2*Precision*Recall}{Precision+Recall} \\
&]= \frac{2*TP}{2*TP+FP+FN}
\end{split}
\end{equation*}

\end{minipage}% <---------------- Note the use of "%"
\begin{minipage}[t]{.33333\textwidth}


\textbf{Principal Components}
\\\\

\begin{equation*}
	V_M = \sum_{m=1}^{M}\lambda_m \; (10.24)
\end{equation*}

\textbf{GMM}
\begin{equation*}
\begin{split}
p(x|\theta) = \sum_{k=1}^K \pi_k \mathcal{N}(x|\mu_k,\Sigma_k)\; (11.3)\\
0 \le \pi_k \le 1, \sum_{k=1}^K \pi_k = 1\;(11.4)
\end{split}
\end{equation*}

\textbf{Support Vector Kernels}
\\\\
Kernel trick
\\\\

\begin{equation*}
\text{hinge} = max(0, 1 - y \cdot \hat{y})
\end{equation*}

\textbf{Information Theory}

% Negative entropy - Example 7.3 p237

\begin{equation*}
	f(x) = x\,log_2\,x
\end{equation*}

\textbf{Misc}

\begin{equation*}
	t = \frac{\bar{X} - \mu}{\frac{\hat{\sigma}}{\sqrt{n}}}
\end{equation*}


\begin{equation*}
	\text{Gini} = \sum_{k=1}^{K}\hat{p}_{mk}(1-\hat{p}_{mk})
\end{equation*}

\begin{equation*}
	\text{CE} = -\sum_{k=1}^{K}\hat{p}_{mk}\log _2 \hat{p}_{mk} ??
\end{equation*}

%

\begin{equation*}
KL(\hat{y} || y) = \sum_{c=1}^{M}\hat{y}_c \log{\frac{\hat{y}_c}{y_c}}
\end{equation*}

\begin{equation*}
cos(x,y) = \frac{x \cdot y}{|x||y|}
\end{equation*}

\textbf{Distributions}

\begin{equation*}
\begin{split}
f(x) &= \frac{1}{\sigma\sqrt{2\pi}}\exp(-\frac{1}{2\sigma^2}(x-\mu)^2) \\
\\
f(\boldsymbol{x}) &= \frac{1}{\sqrt{(2\pi)^n det(\Sigma)}} \exp(-\frac{1}{2}(\boldsymbol{x}-\boldsymbol{\mu})^T \Sigma^{-1} (\boldsymbol{x}-\boldsymbol{\mu}) \\
\end{split}
\end{equation*}
\end{minipage}


\end{document}
